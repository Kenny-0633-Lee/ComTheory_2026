% !TEX program = xelatex
\documentclass[10pt, aspectratio=169]{beamer}
\usetheme{metropolis}
\metroset{sectionpage=none} % [핵심] 섹션 페이지(간지)를 생성하지 않음
\useoutertheme{metropolis}
\useinnertheme{metropolis}
\usecolortheme{metropolis}
\usefonttheme{professionalfonts} 

\usepackage{kotex}
\usepackage{pgfplots}
\pgfplotsset{compat=1.17}
\usepackage{graphicx, caption, hyperref, fontawesome5}
\usepackage{subcaption}
\usepackage{cleveref} % hyperref보다 나중에 불러와야 함
\usepackage{ulem} % 밑줄 긋기 (취소선 등)
\usepackage{lipsum} % 더미 텍스트 생성
\usepackage{fancybox, tcolorbox} % 박스 생성
\usepackage{tcolorbox}  % 박스 생성
\tcbuselibrary{skins, breakable, theorems}   % 박스 스킨 라이브러리
\usepackage{multicol}
% [수식 및 폰트 설정 - 기존 파일 그대로 유지]
\usepackage{amsmath}
\usepackage[math-style=ISO, bold-style=ISO]{unicode-math}
\usepackage{tikz}
\usetikzlibrary{positioning, shapes.geometric, arrows.meta, calc, fit, backgrounds, shapes.geometric, shadows.blur, shapes.symbols, decorations.pathmorphing, decorations.text}
\usefonttheme{professionalfonts}


%%%%%%%%%%%%%%%%%%%%%%%%%%%%%%%%%%%%%%%%%%%%%%%%%%%%
% 시스템 폰트가 아니라, ./fonts 폴더의 파일을 직접 사용합니다.
% 1. 영문/숫자 -> Fira Sans (Small Caps 지원)
\usepackage{fontspec}

% ---------------------------------------------------------
% [1] 영문/숫자 폰트 설정 (Fira Sans)
% ---------------------------------------------------------
% SmallCapsFont 옵션을 명시하지 않아도 OpenType 기능으로 자동 지원하지만,
% 확실하게 작동하도록 설정합니다.
\setmainfont{FiraSans-Regular.ttf}[
    Path = ../../fonts/,
    BoldFont = FiraSans-Bold.ttf,
    ItalicFont = FiraSans-Italic.ttf, % (있다면 추가, 없으면 생략 가능)
    AutoFakeSlant = 0.2
]

\setsansfont{FiraSans-Regular.ttf}[
    Path = ../../fonts/,
    BoldFont = FiraSans-Bold.ttf,
    ItalicFont = FiraSans-Italic.ttf,
    AutoFakeSlant = 0.2
]

% ---------------------------------------------------------
% [2] 한글 폰트 설정 (Noto Sans KR) - 한글에만 적용됨
% ---------------------------------------------------------
% \setmainhangulfont, \setsanshangulfont 명령은
% xetexko 패키지나 kotex 패키지를 사용할 때 유효합니다.
\setmainhangulfont{NotoSansKR-Regular.ttf}[
    Path = ../../fonts/,
    BoldFont = NotoSansKR-Bold.ttf,
    AutoFakeSlant = 0.2
]

\setsanshangulfont{NotoSansKR-Regular.ttf}[
    Path = ../../fonts/,
    BoldFont = NotoSansKR-Bold.ttf,
    AutoFakeSlant = 0.2
]

% 수식 폰트 설정
\setmathfont{Fira Math}
\setmathfont[range={up, bfup}, Scale=MatchLowercase, Path = ../../fonts/]{NotoSansKR-Regular.ttf}
\setmathfont[range={it, bfit}, Scale=MatchLowercase, FakeSlant=0.2, Path = ../../fonts/]{NotoSansKR-Regular.ttf}
% [Font Settings] Project-Local Fonts (Perfect Stability)

% [중요] 그림 경로 수정 (슬라이드 폴더 기준 2단계 위로)
% \graphicspath 명령은 중괄호 {} 안에 경로들을 나열하는 방식입니다.
% 마지막에 반드시 슬래시(/)를 붙여야 합니다.
\graphicspath{
    {./}                            % 1. 현재 tex 파일과 같은 폴더
    {../../assets/}                 % 2. 직접 만든 공용 벡터 이미지 (TikZ 결과물 등)
    {../../assets_extracted/ch02/}  % 3. (중요) 이 챕터의 자동 추출 이미지 폴더
}
%%%%%%%%%%%%%%%%%%%%%%%%%%%%%%%%%%%%%%%%%%%%%
% 이제 Fira Sans가 scshape를 지원하므로 정상 작동합니다.
% 1. 캡션 번호 활성화
\setbeamertemplate{caption}[numbered]
% 2. 캡션 폰트 크기 설정
\setbeamerfont{caption}{size=\scriptsize,shape=\itshape}
\setbeamerfont{section}{size=\footnotesize,shape=\scshape}
\setbeamerfont{subsection}{size=\scriptsize,shape=\itshape}
\setbeamerfont{section in toc}{shape=\scshape}

% 1. 제목 페이지(Title Page)의 메인 제목 설정
\setbeamerfont{subtitle}{shape=\itshape,size=\normalsize}
\setbeamerfont{title}{shape=\upshape,size=\huge}
% 2. 일반 슬라이드의 제목(Frame Title) 설정
\setbeamerfont{frametitle}{shape=\scshape, series=\mdseries}
% (선택사항) 섹션 페이지의 제목도 바꾸고 싶다면 아래 줄 추가
\setbeamerfont{section title}{shape=\scshape}
%%%%%%%%%%%%%%%%%%%%%%%%%%%%%%%%%%%%%%%%%%%%%%
%%%%%%%%%%%%%%%%%%%%%%%%%%%%%%%%%%%%%%%%%%%%%%
% 1. chapter라는 새로운 카운터 생성
\newcounter{chapter}
% 2. 챕터 번호 설정 (여기서 숫자를 바꾸면 됩니다)
\setcounter{chapter}{2}

% [Chapter Numbering] 현재 문서를 "Chapter 2"로 설정
% 섹션 번호 모양을 "2.1", "2.2" 형태로 강제 변경
\renewcommand{\thesection}{\thechapter.\arabic{section}}
% (선택사항) 서브섹션은 자동으로 "2.1.1"이 되지만, 확실하게 하려면 아래 줄도 추가
%\renewcommand{\thesubsection}{\thesection.\arabic{subsection}}
% Figure 2.x 형태로 출력 (chapter 번호 기준)
\renewcommand{\thefigure}{\thechapter.\arabic{figure}}
% Equation 2.x 형태로 출력 (chapter 번호 기준)
\renewcommand{\theequation}{\thechapter.\arabic{equation}}
\renewcommand{\figurename}{Fig.}

% --- [1] 본문 참조 설정 (\cref) ---
% \cref{label}을 썼을 때 "Fig. 번호"로 나오게 설정
% 형식: \crefname{타입}{단수형}{복수형}
\crefname{figure}{Fig.}{Figs.}
\Crefname{figure}{Fig.}{Figs.} % 문장 맨 앞 \Cref 용도

% --- [1] 번호 매기기 깊이 설정 ---
% 0: Chapter, 1: Section, 2: Subsection, 3: Subsubsection
% '1'로 설정하면 Section까지만 번호를 붙이고, Subsection부터는 번호를 생성하지 않습니다.
\setcounter{secnumdepth}{1} 

% --- [2] 목차(TOC) 스타일 설정 ---
% 2-1. 섹션: 번호 표시 (예: 1. Introduction)
\setbeamertemplate{section in toc}[sections numbered]
\setbeamertemplate{subsection in toc}{%
  \leavevmode
  \leftskip=1.0em
  {\color{gray}\textbullet} % 점을 회색으로 은은하게
  \hspace{0.5em}            % 적당한 간격
  {\footnotesize\inserttocsubsection}\par
}

% --- 매크로 정의 ---
% 사용법: \incfig{파일명(확장자제외)}{너비비율(0.0~1.0)}{캡션}
\newcommand{\incfig}[4]{%
    \begin{figure}
        \centering
        % 확장자를 생략하면 graphicspath에서 찾은 파일의 형식을 자동 인식합니다.
        \includegraphics[width=#2\textwidth]{#1} 
        \caption{#3}\label{#4}
    \end{figure}
}
% Preamble에 추가
\newcommand{\fakecaption}[2]{% #1: 참조할 라벨, #2: 텍스트
  \par\vspace{2pt}
  {\usebeamerfont{caption name}\usebeamercolor[fg]{caption name}\figurename~\ref{#1}:}%
  \hspace{1.0em}
  {\usebeamerfont{caption}\usebeamercolor[fg]{caption}\itshape #2}%
}

% 박스 매크로 정의
\newtcbox{\xmybox}[1][red]{on line,
arc=7pt,colback=#1!10!white,colframe=#1!50!black,
before upper={\rule[-3pt]{0pt}{10pt}},boxrule=1pt,
boxsep=0pt,left=6pt,right=6pt,top=2pt,bottom=2pt}

% --- [Colors] Starlink 전용 컬러 팔레트 정의 ---
\definecolor{space_bg}{HTML}{0B0E14}      % 배경: Deep Dark Navy
\definecolor{earth_top}{HTML}{1C2541}     % 지구 상단
\definecolor{earth_bot}{HTML}{000000}     % 지구 하단
\definecolor{starlink_cyan}{HTML}{00F0FF} % 레이저 링크 (Cyan Neon)
\definecolor{signal_green}{HTML}{00FF9D}  % LTE 신호 (Green Neon)
\definecolor{signal_orange}{HTML}{FF9E00} % Ku 신호 (Orange Neon)

% --- [Preamble] TikZ 스타일 전역 정의 (여기에 추가하세요) ---
\tikzset{
    % 빔 효과 (파라미터 #1을 여기서 안전하게 정의)
    beam_cone/.style={
        shade, 
        top color=#1, 
        bottom color=transparent!100, 
        shading angle=0, 
        opacity=0.7
    },
    % 레이저/선 효과
    glow_line/.style={
        line width=2pt, 
        color=#1, 
        opacity=0.8, 
        cap=round
    }
}
%%%%%%%%%%%%%%%%%%%%%%%%%%%%%%%%%%%%%%%%%%%%%%
%%%%%%%%%%%%%%%%%%%%%%%%%%%%%%%%%%%%%%%%%%%%%%

\subtitle{Communication Theory - 2026}
\title{Orientation}
\date{\today}
\author{
    이 경 근 \\
    {\tiny
        % \texorpdfstring{문서에 보일 내용}{PDF 속성에 들어갈 텍스트(특수문자 제외)}
        \texorpdfstring{\raisebox{-0.1ex}{\scalebox{0.85}{\faEnvelope}}}{} \href{mailto:infosec@knu.ac.kr}{infosec@knu.ac.kr} \quad 
        \texorpdfstring{\scalebox{0.9}{\faLinkedin}}{} \scalebox{0.9}{\href{https://www.linkedin.com/in/Kenny-0633-Lee}{Kenny-0633-Lee}}
    }
}
\institute{EE / KNU}

%%%%%%%%%%%%%%%%%%%%%%%%%%%%%%%%%%%%%%%%%%%%%%%
%%%%%%%%%%%%%%%%%%%%%%%%%%%%%%%%%%%%%%%%%%%%%%%
%%%%%%%%%%%%%%%%%%%%%%%%%%%%%%%%%%%%%%%%%%%%%%%

\begin{document}

% 1. 표지
\begin{frame}[plain]
    \titlepage
\end{frame}

% %% 2. 강의 목차
% \begin{frame}{Table of Contents}
%     \large
% %    \begin{multicols}{2} % 자동으로 2컬럼 배분
%         \tableofcontents[hideallsubsections]
% %    \end{multicols}
% \end{frame}

\section{Why Communication Theory matters?}
\begin{frame}{Why Communication Theory Matters?}
    \begin{itemize}
        \setlength\itemsep{1.5em} % 항목 간 간격 조정
        
        % 1. 문명의 기반 (Connectivity)
        \item \textbf{The Invisible Backbone of Modern Civilization} \\
        \small It connects billions of people and devices, enabling the Internet, global economy, and social networks (e.g., Starlink, 5G/6G).
        
        % 2. 물리적 한계 극복 (Reliability)
        \item \textbf{Order from Chaos: Overcoming Noise} \\
        \small It provides the mathematical tools to extract reliable information from noisy, unreliable physical channels (Shannon's Limit).
        
        % 3. 자원 최적화 (Efficiency)
        \item \textbf{Mastering Scarcity: Spectrum \& Energy} \\
        \small It teaches us how to maximize data rates (bits/s) within limited frequency bandwidth (Hz) and power constraints.
        
        % 4. 미래 기술의 언어 (Future)
        \item \textbf{Enabler of Future Technologies} \\
        \small Without advanced communication, there is no AI, IoT, Autonomous Driving, or Quantum Computing.
    \end{itemize}
\end{frame}


% ==============================================================================
% Section: Frequency Allocation (Detailed View)
% ==============================================================================
\section{Frequency Allocation Details}
\begin{frame}
\frametitle{우리나라의 주요 주파수 대역별 용도}
    \begin{figure}
        \centering
        \includegraphics[width=\textwidth, height=0.9\textheight, keepaspectratio]{freq_distribution}
        \caption{대한민국 주파수 분배현황(\protect\href{https://www.spectrum.or.kr/bbs/board.php?bo_table=frequency&page=}{Source: 전파진흥원 Spectrum Portal})}
    \end{figure}

\end{frame}

% ------------------------------------------------------------------------------
% Slide 1: 저주파 대역 (VLF ~ MF)
% ------------------------------------------------------------------------------
\begin{frame}{1/3. Low Frequency Bands (VLF $\sim$ MF)}
    % [상단] 그림을 화면 너비에 꽉 차게 배치 (높이 조절 가능)
    \begin{figure}
        \centering
        \includegraphics[width=\textwidth, height=0.45\textheight, keepaspectratio]{freq_part1}
    \end{figure}
    
    \vspace{0.5em}
    
    % [하단] 설명 (2컬럼으로 나누어 가독성 확보)
    \footnotesize
    \begin{columns}[T] % T: Top alignment (글자 위쪽 정렬)
        % 왼쪽 컬럼: 주요 용도
        \begin{column}{0.48\textwidth}
            \textbf{\textsc{Key Services}}
            \begin{itemize}
                \item \textbf{VLF/LF}: 해상/잠수함 통신, 항공 무선 표지 (Beacon)
                \item \textbf{MF}: \alert{AM Radio} (535$\sim$1605 kHz), 해상 이동 업무
            \end{itemize}
        \end{column}
        
        % 오른쪽 컬럼: 전파 특성
        \begin{column}{0.48\textwidth}
            \textbf{\textsc{Characteristics}}
            \begin{itemize}
                \item \textbf{지표파(Ground Wave)}: 지표면을 따라 멀리 전파됨
                \item 회절성이 매우 강함 (산/건물 통과 유리)
                \item 안테나 크기가 매우 커야 함 ($\lambda$가 길기 때문)
            \end{itemize}
        \end{column}
    \end{columns}
\end{frame}



% ------------------------------------------------------------------------------
% Slide 2: 중/고주파 대역 (HF ~ VHF)
% ------------------------------------------------------------------------------
\begin{frame}{2/3. Mid-Range Frequency (HF $\sim$ VHF)}
    % [상단] 그림
    \begin{figure}
        \centering
        \includegraphics[width=\textwidth, height=0.45\textheight, keepaspectratio]{freq_part2}
    \end{figure}
    
    \vspace{0.5em}
    
    % [하단] 설명
    \footnotesize
    \begin{columns}[T]
        \begin{column}{0.48\textwidth}
            \textbf{\textsc{Key Services}}
            \begin{itemize}
                \item \textbf{HF}: 단파 방송, 아마추어 무선(HAM), 비상 통신
                \item \textbf{VHF}: \alert{FM Radio} (88$\sim$108 MHz), 지상파 DMB, 업무용 무전기
            \end{itemize}
        \end{column}
        
        \begin{column}{0.48\textwidth}
            \textbf{\textsc{Characteristics}}
            \begin{itemize}
                \item \textbf{Sky Wave (HF)}: 전리층 반사를 이용한 원거리(국제) 통신
                \item \textbf{Line-of-Sight (VHF)}: 직진성 시작, 장애물 영향 받음
                \item 음질이 좋고 잡음에 강해짐
            \end{itemize}
        \end{column}
    \end{columns}
\end{frame}

% ------------------------------------------------------------------------------
% Slide 3: 고주파 대역 (UHF ~ SHF)
% ------------------------------------------------------------------------------
\begin{frame}{3/3. High Frequency Bands (UHF $\sim$ EHF)}
    % [상단] 그림
    \begin{figure}
        \centering
        \includegraphics[width=\textwidth, height=0.45\textheight, keepaspectratio]{freq_part3}
    \end{figure}
    
    \vspace{0.5em}
    
    % [하단] 설명
    \footnotesize
    \begin{columns}[T]
        \begin{column}{0.48\textwidth}
            \textbf{\textsc{Key Services (Most Active)}}
            \begin{itemize}
                \item \textbf{UHF}: \alert{Mobile (LTE/5G)}, Wi-Fi(2.4G), GPS, DTV
                \item \textbf{SHF/EHF}: Satellite, Radar, 5G mmWave, Wi-Fi 6E/7
            \end{itemize}
        \end{column}
        
        \begin{column}{0.48\textwidth}
            \textbf{\textsc{Characteristics}}
            \begin{itemize}
                \item 빛에 가까운 강한 직진성
                \item \textbf{Wide Bandwidth}: 초고속 대용량 데이터 전송 가능
                \item 장애물(비, 벽)에 의한 감쇄 심함
            \end{itemize}
        \end{column}
    \end{columns}
\end{frame}


% ========================================================================
% Starlink : Tikz diagram
% ========================================================================
\begin{frame}{Starlink Network Architecture}
    \centering
    \resizebox{0.95\textwidth}{!}{%
        \begin{tikzpicture}[
            font=\sffamily,
            background rectangle/.style={fill=space_bg, rounded corners=8pt}, 
            show background rectangle
        ]
            % --- [레이아웃 변수 정의] ---
            \def\H{4.5}
            \def\R{14}
            
            % [수정됨] 변수명에서 언더스코어(_) 제거 (에러 해결)
            \def\YSide{-0.75}  % 양 끝 지표면 높이
            \def\YCenter{0}    % 중앙 지표면 높이

            % === 1. 지구 (Earth Horizon) ===
            \begin{scope}
                \clip (-7, -3.5) rectangle (7, \H+1);
                \shade[top color=earth_top, bottom color=earth_bot] (0, -\R) circle (\R);
                \draw[white!10, dashed, ultra thin] (0, -\R) circle (\R+0.05);
                \node[text=white!40, font=\scshape\scriptsize, align=center] at (0, -2.0) {Earth Surface};
            \end{scope}

            % === 2. 위성 (Satellites) ===
            \tikzset{
                sat_pic/.pic={
                    \fill[gray!50!blue, blur shadow={shadow blur steps=3}] (-0.6, -0.2) rectangle (0.6, 0.2);
                    \draw[white!40, ultra thin] (-0.6, -0.2) grid[xstep=0.2, ystep=0.4] (0.6, 0.2);
                    \fill[white!90!gray] (-0.2, -0.05) rectangle (0.2, 0.05);
                    \node[text=black, scale=0.4] at (0,0) {\faSatellite};
                }
            }
            \coordinate (S1) at (-3.5, \H);
            \coordinate (S2) at (3.5, \H);
            \path (S1) pic {sat_pic};
            \path (S2) pic {sat_pic};
            \node[text=white, font=\scshape\scriptsize, above=0.3cm of S1] {Starlink V2 (LEO)};
            \node[text=white, font=\scshape\scriptsize, above=0.3cm of S2] {Starlink V2 (LEO)};

            % === 3. 지상 장비 (Ground Segment) ===
            % [수정됨] \YSide, \YCenter 사용
            
            % (A) 스마트폰
            \node[text=white, scale=1.5, anchor=south] (phone) at (-4.5, \YSide) {\faMobile*};
            \node[text=signal_green, font=\scshape\tiny, below=0.1cm of phone] {Direct to Cell};
            \node[text=white!60, font=\tiny, align=center, below=0.35cm of phone] {Smartphone\\(LTE Band)};

            % (B) 사용자 단말
            \node[text=white, scale=1.2, anchor=south] (dish) at (-0.5, \YCenter) {\faSatelliteDish};
            \node[text=signal_orange, font=\scshape\tiny, below=0.1cm of dish] {User Terminal (Dishy)};
            \node[text=white!60, font=\tiny, align=center, below=0.35cm of dish] {Home/Biz\\(Ku-band)};

            % (C) 게이트웨이
            \node[text=white, scale=1.2, anchor=south] (gateway) at (4.5, \YSide) {\faBroadcastTower};
            \node[text=starlink_cyan, font=\scshape\tiny, below=0.1cm of gateway] {Gateway};
            \node[text=white!60, font=\tiny, align=center, below=0.35cm of gateway] {Backhaul\\(Ka/E-band)};

            % === 4. 통신 링크 ===
            % (1) Laser Link
            \draw[glow_line=starlink_cyan] (S1) -- (S2); 
            \fill[starlink_cyan, opacity=0.2] ($(S1)+(0,-0.05)$) rectangle ($(S2)+(0,0.05)$);
            \node[text=starlink_cyan, font=\scshape\tiny, fill=space_bg, inner sep=2pt, rounded corners] at (0, \H) {Optical Laser Link};

            % (2) Smartphone Beam
            \path[beam_cone=signal_green] 
                ($(S1)+(-0.2,-0.2)$) -- ($(phone.north)+(-0.3,0)$) -- ($(phone.north)+(0.3,0)$) -- cycle;
            \node[text=signal_green, font=\tiny, rotate=-102] at (-4.35, 2.5) {1.9GHz};

            % (3) User Dish Beam
            \path[beam_cone=signal_orange] 
                ($(S1)+(0.2,-0.2)$) -- ($(dish.north)+(-0.4,0)$) -- ($(dish.north)+(0.4,0)$) -- cycle;
            \node[text=signal_orange, font=\tiny, rotate=-48] at (-1.5, 2.5) {Ku-band (12/14GHz)};

            % (4) Gateway Beam
            \path[beam_cone=starlink_cyan] 
                ($(S2)+(0,-0.2)$) -- ($(gateway.north)+(-0.4,0)$) -- ($(gateway.north)+(0.4,0)$) -- cycle;
            \node[text=starlink_cyan, font=\tiny, rotate=-80] at (3.6, 2.5) {Ka/E-band (20-30GHz)/(70-80GHz)};
        \end{tikzpicture}
    }
\end{frame}


% 5G Network
\begin{frame}{5G Terrestrial Network Architecture (PLMN)}
    \centering
    \resizebox{0.95\textwidth}{!}{%
        \begin{tikzpicture}[
            font=\sffamily,
            % [수정 1] 배경을 흰색으로 변경
            background rectangle/.style={fill=white}, 
            show background rectangle,
            node distance=1.5cm and 2cm
        ]
            % --- [스타일 정의] ---
            \tikzset{
                % Core Network Nodes (가독성을 위해 연한 파스텔톤 배경 + 검은 글씨)
                control_node/.style={
                    rectangle, rounded corners=3pt, draw=blue!80!black, thick,
                    fill=blue!5, text=black, align=center,
                    minimum width=2.2cm, minimum height=1.2cm,
                    drop shadow
                },
                user_node/.style={
                    rectangle, rounded corners=3pt, draw=green!60!black, thick,
                    fill=green!5, text=black, align=center,
                    minimum width=2.2cm, minimum height=1.2cm,
                    drop shadow
                },
                ran_node/.style={
                    rectangle, rounded corners=3pt, draw=orange!80!black, thick,
                    fill=orange!5, text=black, align=center,
                    minimum width=1.8cm, minimum height=3cm,
                    drop shadow
                },
                % [수정 2] 무선 구간 (Air Interface) - 파동 모양(Snake)
                wireless_link/.style={
                    decorate, decoration={snake, amplitude=.5mm, segment length=2.5mm, post length=1mm},
                    draw=blue!70!black, thick, ->, >=stealth
                },
                % [수정 2] 유선 구간 - User Plane (실선)
                wired_user/.style={
                    draw=green!50!black, thick, ->, >=stealth
                },
                % [수정 2] 유선 구간 - Control Plane (점선)
                wired_ctrl/.style={
                    draw=blue!50!black, thick, dashed, ->, >=stealth
                },
                % 라벨 스타일 (흰 배경을 깔아서 선과 겹쳐도 잘 보이게)
                label_box/.style={
                    fill=white, inner sep=1.5pt, text=black, font=\tiny\bfseries, opacity=0.9
                }
            }

            % === 1. User Equipment (UE) & Frequency Bands ===
            % 왼쪽 배치
            
            % (A) Smartphone (Sub-6 GHz 주력)
            \node[text=black, scale=1.5] (phone) at (-5, -2) {\faMobile*};
            \node[text=black, font=\scriptsize, below=0.1cm of phone] {Smartphone};
            
            % (B) V2X Car (URLLC, Sub-6 or mmWave)
            \node[text=black, scale=1.3] (car) at (-5, 0) {\faCar};
            \node[text=black, font=\scriptsize, below=0.1cm of car] {V2X Car};
            
            % (C) Smart Factory (mmWave 주력 - 초고속/저지연)
            \node[text=black, scale=1.3] (factory) at (-5, 2) {\faIndustry};
            \node[text=black, font=\scriptsize, below=0.1cm of factory] {Smart Factory};


            % === 2. RAN (gNodeB) ===
            \node[ran_node] (gnb) at (-1, 0) {
                \huge \faBroadcastTower \\[0.3em]
                \textbf{gNodeB} \\
                \tiny (5G NR)
            };


            % === 3. 5G Core (Control & User Plane) ===
            
            % AMF (Access & Mobility) - Top Left of Core
            \node[control_node] (amf) at (3, 2.0) {\textbf{AMF} \\ \tiny Access Control};
            
            % SMF (Session Mgmt) - Top Right of Core
            \node[control_node] (smf) at (6.5, 2.0) {\textbf{SMF} \\ \tiny Session Mgmt};
            
            % UPF (User Plane) - Bottom Center of Core
            \node[user_node] (upf) at (4.75, -2) {\textbf{UPF} \\ \tiny Packet Routing};


            % === 4. Data Network (Internet) ===
            \node[cloud, draw=gray, thick, fill=gray!5, cloud puffs=11, minimum width=2.5cm, minimum height=1.8cm] (dn) at (8.5, -2) {};
            \node[text=black, align=center, font=\small\bfseries] at (dn) {Internet\\(DN)};


            % === 5. Links (Connections) ===
            
            % --- [Wireless Links (Air Interface)] ---
            % [수정 3] 통신 밴드 명시 (FR1, FR2)
            
            % Phone -> gNB (Sub-6 GHz)
            \draw[wireless_link] (phone) -- (gnb.south west) 
                node[midway, label_box, rotate=20] {3.5 GHz (FR1)};
                
            % Car -> gNB (V2X band)
            \draw[wireless_link] (car) -- (gnb.west)
                node[midway, label_box] {5.9 GHz (C-V2X)};
                
            % Factory -> gNB (mmWave)
            \draw[wireless_link] (factory) -- (gnb.north west)
                node[midway, label_box, rotate=-20] {28 GHz (FR2)};


            % --- [Wired Links (Backhaul)] ---
            % [수정 4] 유선망은 광케이블(Optical)임
            
            % N2: gNB -> AMF (Control)
            \draw[wired_ctrl] (gnb.north east) -- (amf.west) 
                node[midway, label_box, sloped] {N2 (Optical)};
            
            % N3: gNB -> UPF (User Data)
            \draw[wired_user] (gnb.south east) -- (upf.west)
                node[midway, label_box, sloped] {N3 (Optical)};
                
            % N4: SMF -> UPF (Control)
            \draw[wired_ctrl] (smf.south) -- (upf.north -| smf.south)
                node[midway, label_box] {N4};
                
            % SBA: AMF <-> SMF
            \draw[wired_ctrl, <->] (amf) -- (smf)
                node[midway, label_box] {SBA (HTTP/2)};
                
            % N6: UPF -> Internet
            \draw[wired_user] (upf) -- (dn)
                node[midway, label_box] {N6 (Backbone)};


            % === 6. 구획 표시 (Grouping) ===
            \begin{scope}[on background layer]
                % RAN Area Label
                \node[text=gray, font=\bfseries\scriptsize, above=0.1cm of gnb] {Radio Access Network};
                
                % Core Area Box
                \draw[dashed, gray!40, rounded corners=10pt] (1.5, -3.2) rectangle (8.0, 3.2);
                \node[text=gray, font=\bfseries\scriptsize] at (4.75, 2.9) {5G Core Network (5GC)};
            \end{scope}

        \end{tikzpicture}
    }
\end{frame}


% Sonar 통신 아키텍쳐
\begin{frame}{Underwater Communication: SONAR System Architecture}
    % figure 환경 시작
    \begin{figure}[b]
        \centering
        % 이미지 삽입
        \includegraphics[height=0.65\textheight, keepaspectratio]{fig_sonar_system.png}
        
        % 캡션 추가 (그림 제목)
        \caption{General Architecture of SONAR System}
        
        % (선택사항) 캡션에 라벨을 달아두면 나중에 \ref{fig:sonar}로 참조 가능
        \label{fig:sonar}
    \end{figure}
    
    % 출처는 캡션과 별도로 하단에 표시 (디자인적 선택)
    \vfill
    \begin{flushright}
        \tiny \color{gray}
        Source: \href{https://www.mdpi.com/2077-1312/11/7/1279}{MDPI J. Mar. Sci. Eng. 2023, 11(7), 1279}
    \end{flushright}
\end{frame}


% Summary
\section*{Summary}
\begin{frame}{Summary}
\centering

{\huge \textbf{``Communication Theory is the \textcolor{red}{Foundation} of our Connected Future.''}}
\end{frame}

\end{document}